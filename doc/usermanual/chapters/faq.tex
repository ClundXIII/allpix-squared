\section{Frequently Asked Questions}
\label{sec:faq}
\begin{description}[style=nextline]
\item[How do I run a module only for one detector?]
This is only possible for detector modules (which are constructed to work on individual detectors). To run it on a single detector one should add a parameter \texttt{name} specifying the name of the detector (as given in the detector configuration file).
\item[How do I run a module only for a specific detector type?]
This is only possible for detector modules (which are constructed to work on individual detectors). To run it for a specific type of detectors one should add a parameter \texttt{type} with the type of the detector model (as given in the detector configuration file by the \texttt{model} parameter).
\item[How can I run the exact same type of module with different settings?]
This is possible by using the \texttt{input} and \texttt{output} parameters of a module that specialize the location where the messages from the modules are send to and received from. By default both the input and the output of module defaults to the message without a name.
\item[How can I temporarily ignore a module during development?]
The section header of a particular module in the configuration file can be replaced by the string \texttt{Ignore}. The section and all of its key/value pairs are then ignored.
\item[Can I get a high verbosity level only for a specific module?]
Yes, it is possible to specify verbosity levels and log formats per module. This can be done by adding a \texttt{log\_level} and/or \texttt{log\_format} key to a specific module to replace the parameter in the global configuration sections.
\item[I want to use a detector model with one or several small changes, do I have to create a whole new model for this?]
No, models can be specialized in the detector configuration file. This feature is available to, for example, use models with different sensor thicknesses. To specialize a detector model the key that should be changed in the standard detector model (like \texttt{sensor\_thickness}) should be added as key to the section of the detector configuration (which is always required to already contain the position, orientation and the base model). Only parameters in the header of detector models can be changed. If support layers should be changed, or new support layers are needed, a new model should be created instead.
\item[How do I access the history of a particular object?]
Many objects can have a link to related other objects (for example \texttt{getPropagatedCharges} in the \texttt{PixelCharge} object), containing the history of the object (thus the objects that were used to construct the current object). These referenced objects are stored as special ROOT pointers inside the object, which can only be accessed if the referenced object is available in memory. In \apsq this requirement can be automatically fullfilled by also binding the history object in a module, assuming the creating module actually saved the history with the object which is not strictly required. During analysis the tree holding the referenced object should be loaded and pointing to the same event entry as the object that request the reference to load it. If the referenced object can not be loaded an exception is required to be thrown by the retrieving method.
\item[How do I access the Monte Carlo truth of a specific PixelHit?]
The Monte Carlo truth is just part of the indirect history of a PixelHit. This means that the Monte-Carlo truth can be fetched as described in the question above. However take notice that there are multiple layers between a PixelHit and its MCParticles, which are the PixelCharge, PropagatedCharges and DepositedCharges. These should all be loaded in memory to make it possible to fetch the history. Because getting the Monte Carlo truth of a PixelHit is quite a common thing a \texttt{getMCParticles} convenience method is available which searches all the layers of the history and returns an exception if any of the in between steps is not available or not loaded.
\item[Can I import an electric field from TCAD and use that for simulating propagation?]
Yes, the framework includes a tool to convert DF-ISE files from TCAD to an internal format which \apsq can parse. More information about this tool can be found in Section \ref{sec:tcad_electric_field_converter}, instructions to import the generated field are given in Section \ref{sec:module_electric_field}.
\end{description}

\todo{Add more questions}
