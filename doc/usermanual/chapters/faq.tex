\section{Frequently Asked Questions}
\label{sec:faq}
\begin{description}[style=nextline]
\item[How do I run a module only for one detector?]
This is only possible for detector modules (which are constructed to work on individual detectors). To run it on a single detector one should add a parameter \texttt{name} having the name of the detector (from the specified in the detector configuration file) as the key.
\item[How do I run a module only for a specific detector type?]
This is only possible for detector modules (which are constructed to work on individual detectors). To run it for a specific type of detectors one should add a parameter \texttt{type} with the type of the detector model (specified in the detector configuration file as the \texttt{model} parameter) as its key.
\item[How can I temporarily ignore a module during development?]
The section header of a particular module in the configuration file can be replaced by the string \texttt{Ignore}. The section and all of its key/value pairs are then ignored.
\item[Can I get a high verbosity level only for a specific module?]
Yes it is possible to specify verbosity levels and log formats per module. This can be done by adding a \texttt{log\_level} and/or \texttt{log\_format} key to a specific module instead of adding in to one of the global configuration sections.
\end{description}

\todo{Add more questions}
