\section{Module \& Detector Development}
\subsection{Implementing a New Module}
\label{sec:building_new_module}

\todo{add description of what should be a module and what not (aka analysis)}
Before creating a module it is essential to read through the framework module manager documentation in Section \ref{sec:module_manager}, the information about the directory structure in Section \ref{sec:module_files} and the details of the module structure in Section \ref{sec:module_structure}. Thereafter the steps below should provide enough details for starting with a new module \texttt{ModuleName} (constantly replacing  \texttt{ModuleName} with the real name of the new module):
\begin{enumerate}
\item Run the module initialization script at \textit{etc/scripts/make\_module.sh} in the repository. The script will ask for the name of the model and the type (unique or detector-specific). It creates the directory with a minimal example toe get started together with a setup of the documentation in \textit{README.md}.
\item Before continuing to implement the module it is recommended to check and update the introductory documention in \textit{README.md}. Also the Doxygen documentation in \textit{\texttt{ModuleName}.hpp} can be extended to give a basic description of the module.
\item Now the constructor, and possibly the \texttt{init}, \texttt{run} and/or \texttt{finalize} methods can be written, depending on what the new module needs.
\end{enumerate}

After this, it is up to the developer to implement all the required functionality in the module. Keep considering however that at some point it may be beneficial to split up modules to support the modular design of \apsq. Various sources which may be primarily useful during the development of the module include:
\begin{itemize}
\item The framework documentation in Section \ref{sec:framework} for an introduction to the different parts of the framework.
\item The module documentation in Section \ref{sec:modules} for a description of functionality other modules already provide and to look for similar modules which can help during development.
\item The Doxygen (core) reference documentation included in the framework \todo{available at location X}.
\item The latest version of the source code of all the modules (and the core itself). Freely available to copy and modify under the MIT license at \url{https://gitlab.cern.ch/simonspa/allpix-squared/tree/master}.
\end{itemize}

Any module that may be useful for other people can be contributed back to the main repository. It is very much encouraged to send a merge-request at \url{https://gitlab.cern.ch/simonspa/allpix-squared/merge_requests}.

\subsection{Adding a New Detector Model}
\label{sec:adding_detector_model}
Custom detector models can be easily added to the framework. Required information, before writing the model, is Section \ref{sec:config_file_format} describing the file format, Section \ref{sec:config_values} for information about the units used in \apsq and the full Section \ref{sec:models_geometry} describing the geometry and detector models. In particular Section \ref{sec:detector_models} explains all the parameters of the detector model. The default models shipped in \textit{models} could serve as examples. To write your own module follow the steps below:
\begin{enumerate}
\item Create a new file with the internal name of the model followed by the \textit{.conf} suffix (for example \texttt{your\_model.conf}).
\item Add a configuration parameter \texttt{type} with the type of the model, at the moment either 'monolithic' or 'hybrid' for respectively monolithic sensors or hybrid models with bump bonds.
\item Add all the required parameters and possibly other optional parameters explained in Section \ref{sec:detector_models}.
\item Include the detector model in the search path of the framework by adding the \texttt{model\_path} parameter to the general setting of the main configuration (see Section \ref{sec:framework_parameters}) pointing to either directly to the detector model file or the detector containing it (note that files in this path overwrite models with the same name in the default model folder).
\end{enumerate}

Models can be contributed to the repository to make them available to other users of the framework. To add the detector model to the framework the configuration file should be moved to the \textit{models} folder of the repository. Then the file should be added to the installation target in the \textit{CMakeLists.txt} file in the \textit{models} directory. Afterwards a merge-request can be created at \url{https://gitlab.cern.ch/simonspa/allpix-squared/merge_requests}.
