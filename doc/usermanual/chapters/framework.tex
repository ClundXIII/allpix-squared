\section{The \apsq Framework}
The application is split up in the following four main components that together form \apsq:
\begin{enumerate}
\item \textbf{Core}: Sometimes referred as the framework itself, with the rest of the components separate. The core, object and tools are also known as the core framework. The core contains the internal logic to initiate the modules, provide the geometry, facilitate module communication and run the event sequence. The core should keep its dependencies to a minimum and remain separated from the other components as far as posibble. This is the main component discussed in this section.
\item \textbf{Modules}: A set of methods that execute a (optional) subset of the simulation chain. These are build as separate libraries, loaded dynamically by the core. The set of available modules are discussed in more detail in \needref.
\item \textbf{Objects}: Objects are sets of data that are passed around between modules. They are contained into a single library and are transferred using the message framework provided by the core. Modules can listen and bind to messages they wish to receive. Messages are passed around by the object type they are carrying, but they can also be named to allow the use of more advanced modules and sophicates simulations setups. Messages are meant to be read-only and a copy of the data should be made if a module wishes to change the data. More information about the messaging system and the currently supported set of objects are found in section \ref{sec:objects_messages}.
\item \textbf{Tools}: \apsq provides a set of header-only 'tools' that can be used by modules and provide access to common logic to interface various library that could be of use for different modules. An example is a Eigen Runge-Kutta solver and a set of template specializations for ROOT and Geant4 configuration. More information about these can be found in \needref. This set of tools is different from the set of core utilities the framework provides by itself, which are part of the core and explained in \ref{sec:logging_utilities}
\end{enumerate}
Finally \apsq provides an executable which instantiates the core, passes the configuration and runs the simulation chain.

In this section first an overview is given about the architectural setup of the core and how it interacts with the total \apsq framework. Afterwards the separate subcomponents are discussed and explained in more detail. Some code will be shown throughout the paragraphs, but users not interested in technical details can skip these.

\subsection{Architecture of the Core}
The core is constructed as a light-weight framework that provides various subsystems to the modules. It also contains the part responsible for instantiating and running the modules from the supplied configuration file. The core is structured around five subsystems of which four are centered around managers and the fifth contain a set of simple general utilities. The systems provided are given below:
\begin{enumerate}
\item \textbf{Configuration}: Provides a general configuration object from which data can be stored and retrieved, together with a TOML-like\needref file parser to instantiate the configurations. Also provides a general \apsq configuration manager giving access to the main configuration file and section. It is used by module manager system to find the required instantiations and access the global configuration. More information is given in \ref{sec:config_parameters}.
\item \textbf{Module}: Contain the base class of all the \apsq modules and the manager responsible for loading and running the modules (using the configuration system above). This component is discussed in more detail in section \ref{sec:module_manager}.
\item \textbf{Geometry}: Supplies helpers for the simulation geometry. The manager contains all the registered detectors. A detector has a certain position and orientation linked to an instantiation of a particular detector model. The detector model contains all the parameters that describe the geometry of the detector. Most modules only need a subset of these parameters, but they are all together used to construct the Geant4 geometry. More details about the geometry and detector models is provided in section \ref{sec:models_geometry}.
\item \textbf{Messenger}: The messenger is responsible for sending objects from one module to another. The messenger object is passed to every module and can be used to bind to messages to listen for. Messages with objects can also be dispatched through the messenger to send data further in the simulation chain. Refer to section \ref{sec:objects_messages} for more details.
\item \textbf{Utilities}: The framework provides a set of simple utilities for logging, file and directory access, random number seeding and unit conversion. An explanation how to use of these utils can be found in section \ref{sec:logging_utilities}. 
\end{enumerate}

A set of base exceptions is also provided in the utilities, which are inherited and extended in the other components. Proper use of exceptions together with logging warnings and reporting errors can make the framework much easier to use and debug. A few notes about the use and the structure of exceptions are given in section \ref{sec:error_reporting_exceptions}.

\subsection{Configuration and Parameters}
\label{sec:config_parameters}
The modules and the framework can be configured through configuration files. An explanation how to use the various configuration files is given in \needref. All configuration files follow the same format, but the framework has a few exceptional rules for accessing the main configuration file.

\subsubsection{File format}
\label{sec:config_file_format}
Throughout the framework a standard format is used for the configuration files, this format is a simplified version of TOML\needref (which is again inspired by JSON and Windows INI files). The rules for this format are as follows:
\begin{enumerate}
\item All whitespace at the beginning or end of a line should be stripped by the parser. Empty lines should be ignored.
\item Every non-empty line should start with either a \texttt{\#}, a \texttt{[} or a alphanumeric character. Every other character should lead to an immediate parse error.
\item If the line starts with a \texttt{\#}, the line is handled as comment and all other content on the same line is ignored
\item If the line starts with a \texttt{[}, the line indicates a section header (also known as configuration header). The line should contain an alphanumeric string indicating the header name followed by an \texttt{]} to end the header (a missing \texttt{]} should raise an exception). Multiple section header with the same name are allowed. All key-value pairs following this section header are part of this section until a new section header is started. After a optional number of ignored whitespace characters there may eventually only be \texttt{\#} character. If that is the case the rest of the line is handled as specified in point 3.
\item If the line starts with an alphanumeric character, the line should indicate a key-value pair. The beginning of the line should contain an alphanumeric string indicating the 'key'. After a optional number of ignored whitespace the key should be followed by an \texttt{\=}. Any text between the \texttt{\=} and the first \texttt{\#} character not enclosed within a pair of \texttt{"} characters is known as the non-stripped 'value'. Any character from the \texttt{"} is handled as specified in point 3. If the line does not contain any non-enclosed \texttt{\#} character the value ends at the end of the line instead. The 'value' of the key-value pair is the non-stripped 'value' with all whitespace in front and the end stripped.
\item The value can either be accessed as a single value or an array. If the value is accessed as an array the string is splitted on every whitespace or \texttt{,} character that is not enclosed in a pair of \texttt{"} characters. All empty entities coming out of the split are ignored. All other entities are threated as single values in the array.
\item All single values are stored as a string containing at least one character. The conversion to the actual type happens on accessing the value.
\item All key-value pairs defined before the first section header are part of the zero-length empty section header
\end{enumerate}

\subsubsection{Accessing parameters}
All values are accessed from the configuration object. If the key is a string called \textbf{key}, the object is named \textbf{config} and the type \textbf{TYPE} is a valid C++ type that the value should represent, the values should be accessed and defaulted by the following methods:
\begin{minted}[frame=single,framesep=3pt,breaklines=true,tabsize=2,linenos]{c++}
// returns true if the key exists and false otherwise
config.has("key") 
// returns the key as the given type, throws an exception if not existing
config.get<TYPE>("key") 
// return the key in the given type or the provided default value if it does not exist
config.getDefault<TYPE>("key", default_value) 
// returns an array of single values of the given type; throws if the key does not exist
config.getArray<TYPE>("key")
// returns the key as literal string
config.getText("key") 
// set the value of key to the default value if the key is not defined
config.setDefault("key", default_value) 
// set the value of the key to the defaults array if key is not defined
config.setDefaultArray<TYPE>("key", vector_of_default_values)
\end{minted}

The conversions to the type are using the from\_string and to\_string methods provided by the string utility library described in section \needref. This largely follows the standard C++ parsing, with one important exception. If (and only if) the value is retrieved as any C/C++ string type and the string is fully enclosed by a pair of \texttt{"} characters they are stripped before returning the value. 

\subsubsection{Special rules for the main configuration}
The main configuration file is provided during construction of the framework. The configuration file contains a set of sections as defined above. There are however a few section names that are ignored if they are not specifically accessed by name (but by the general return-all-configurations method instead). These are the following:
\begin{itemize}
\item The \textbf{global} header sections: These are (by default) all the zero-length section headers (including the one at the start) and all with the header \textbf{AllPix}. These are combined and accessed together as the so-called global configuration, which contain all the parameters of the general framework (see section \ref{sec:framework_parameters} for details). All key-value pairs defined in this section are also merged with all individual configurations as long the key is not already defined in the specific configuration.
\item The \textbf{ignore} header sections: The section with name Ignore is (by default) fully ignored. All key-value pairs as well as the section itself are threated as if they were never defined. These sections are useful for quickly enabling and disabling certain sections.
\end{itemize}

The rest of the normal sections are used to instantiate all the modules as explained in section \ref{sec:module_instantiation}.

\subsection{Modules and the Module Manager}
\label{sec:module_manager}
\apsq is a modular framework, the core idea is to separate functionality in various independent modules. The various modules should be defined in a directory in \textit{src/modules/}. The name of the directory is the unique name of the module. The suggested naming scheme is CamelCase, thus an example module would be \textit{GenericPropagation}. There are two different kind of modules which can be defined:
\begin{itemize}
\item \textbf{Detector}: Modules that are specific to a single detector. They are replicated for all required detectors.
\item \textbf{Unique}: Modules of which only one instance is allowed to exist per run of the framework.
\end{itemize}
The type of module determines the kind of constructor used, the internal unique name and the supported configuration parameters. More details about the instantiation logic for the different kind of modules can be found in \needref.

\subsubsection{Files for a module}
\label{sec:module_files}
Every module directory should at the minimum contain the following documents (with \textit{module\_name} replaced by the name of the module):
\begin{itemize}
\item \textbf{CMakeLists.txt}: The build script to load the dependencies and set the source files
\item \textbf{README.md}: The short documentation of the module
\item \textbf{\textit{module\_name}.tex}: The documentation of the module
\item \textbf{\textit{module\_name}Module.hpp}: The header file of the module (note that another name can be used for this source file, but that is deprecated)
\item \textbf{\textit{module\_name}Module.cpp}: The implementation file of the module
\end{itemize}
The files are discussed in more detail below. Before the module can be used it should also be linked to the main CMakeLists.txt in \textit{src/modules/} by adding \\ ALLPIX\_BUILD\_MODULE(\textit{module\_name}) with the \textit{module\_name} being the directory of the module. 

More information about constructing your own module can be found in section \ref{sec:building_new_module}.

\paragraph{CMakeLists.txt}
Contains the build description of the module with the following components:
\begin{enumerate}
\item On the first line either ALLPIX\_DETECTOR\_MODULE(MODULE\_NAME) \textbf{or} \\ ALLPIX\_UNIQUE\_MODULE(MODULE\_NAME) dependent on the type of the module defined. The internal name of the module is saved to the \$\{MODULE\_NAME\} variable which should be used as argument to the other functions. Another name can be used as well, but below we exclusively use \$\{MODULE\_NAME\}
\item The next lines to load the required dependencies. Only ROOT is automatically added to the include headers and linked in the module.
\item A line with ALLPIX\_MODULE\_SOURCES(\$\{MODULE\_NAME\} \textless sources \textgreater) with sources replaced by all the source files of this module
\item Several lines to include the directories and load the libraries
\item A line containing ALLPIX\_MODULE\_INSTALL(\$\{MODULE\_NAME\} to setup the required target for the module to be installed to.
\end{enumerate}

An example of a simple CMakeLists.txt of a module named \textbf{Test} is the following
\vspace{5pt}

\begin{minted}[frame=single,framesep=3pt,breaklines=true,tabsize=2,linenos]{cmake}
# Define module and save name to MODULE_NAME
ALLPIX_UNIQUE_MODULE(MODULE_NAME) # Replace by ALLPIX_DETECTOR_MODULE(MODULE_NAME) to define a detector module

# Load required dependencies...

# Add sources
ALLPIX_MODULE_SOURCES(${MODULE_NAME} 
    TestModule.cpp
)

# Include directories and link the libraries...

# Provide standard install target
ALLPIX_MODULE_INSTALL(${MODULE_NAME})
\end{minted}

\paragraph{README.md}
\todo

\paragraph{\textit{module\_name}.tex}
\todo

\paragraph{\textit{module\_name}Module.hpp and \textit{module\_name}Module.cpp}
All the modules should inherit from the Module base class which can be found in \textit{src/core/module/Module.hpp}. The module base class provide two base constructors, a few convenient methods and several methods to override. Every module should provide a fixed constructor which is always called during construction by the module instantiation logic. The arguments for the constructor differs for unique and detector modules. For unique modules the empty constructor for a \textbf{Test} module should be:
\begin{minted}[frame=single,framesep=3pt,breaklines=true,tabsize=2,linenos]{c++}
TestModule(Configuration config, Messenger* messenger, GeometryManager* geo_manager) {}
\end{minted}

For unique modules the first two arguments are the same, but the last argument is a shared\_ptr to the linked detector instead. It should also always forward this provided detector to the base class. Thus an empty constructor of a detector module should be:
\begin{minted}[frame=single,framesep=3pt,breaklines=true,tabsize=2,linenos]{c++}
TestModule(Configuration config, Messenger* messenger, std::shared_ptr<Detector> detector): Module(detector) {}
\end{minted}

All the module are thus passed the Configuration object holding the config parameters for that specific object which can be accessed as explained in \needref. Furthermore a pointer to the Messenger is passed which can be used to both bind variables to receive and dispatch messages as explained in \needref. Finally either a pointer to the GeometryManager is passed, which can be used to fetch all the detectors, or a instance of the specific linked detector. The constructor should normally be used to bind the required messages and set configuration defaults. Besides the constructor, every module can override the following methods:
\begin{itemize}
\item init(): Called after loading and constructing all modules and before initializing the run. In case of failure an exception can be thrown from the constructor.
\item run(unsigned int event\_id): Called for every event in the simulation run with the number of the event supplied (starting from one). An exception should be thrown for every serious error, otherwise an warning should be logged.
\item finalize(): Called after running all events in the run, before destructing the module. Any exceptions should be called from here instead of the destructor.
\end{itemize}

\todo more information about provided module functions \todo

\subsubsection{Module instantiation}
\label{sec:module_instantiation}
The modules are dynamically loaded and instantiated by the Module Manager. Modules are constructed, initialized, runned and finalized in the linear order they are defined in the configuration file. Thus the configuration file should follow the order of the real process. For every non-special section in the main configuration file (see \ref{sec:config_parameters} for more details) a corresponding library is searched which should contain the module. A module library should have the name libAllPixModule\textit{module\_name} with \textit{module\_name} replaced with the actual name of the module from the configuration section header. The module search path is as follows:
\begin{enumerate}
\item The modules already loaded before from an earlier section header
\item All directories in the global configuration parameter \textit{library\_directories} in the provided order if this parameter exists
\item The rpath of the executable, that should automatically point to the libraries that are build and installed together with the executable.
\item The other standard locations (after the rpath) as specified in the POSIX standard \needcite.
\end{enumerate}

If the module definition is succesfull it is determined if the module is an unique or a detector module. Depending on the type, the instantiation logic continues as follows:
\begin{itemize}
\item \textbf{Unique}: The name of the module is searched in the list of already registered modules. If the module already exists an exception is raised, otherwise the module is added to the run list.
\item \textbf{Detector}: In this case the modules are instantiated by priority, replacing earlier instantiations with lower priority. Module section that are matched for a specific detector name using the configuration parameter \textit{name} have the highest priority, followed by module sections matched with their model type using \textit{type}. Modules without either the \textit{name}, \textit{type} or both, produce instantiations for all registered modules and have the lowest priority. An exception is raised if a instantiation is requested of the same priority as an earlier one.
\end{itemize}

\subsection{Geometry and Detectors}
\label{sec:models_geometry}
Simulations are frequently run on a set of different detectors (also known as as a telescope). The set of these individual detectors is what AllPix defines as the geometry (note that internal modules are allowed to extend that definition of geometry, but for the framework itself the world only consists of set of independent detectors). Every detector has a set of properties attached to it:
\begin{itemize}
\item A unique \textbf{name} to refer to the detector in the configuration.
\item The \textbf{position} in the world frame. This is the position of the geometric center of the sensitive device (sensor) given in world coordinates as X, Y and Z (note that any additional components like the chip or the PCB are not including for determining the geometric center).
\item The \textbf{orientation} given as Euler angles using the extrinsic Z-X-Z convention relative to the world frame (also known as the 1-3-1 or the "x-convention" and the most widely used definition of Euler angles \needcite). 
\item A \textbf{model} identified by a string named \text{type}. The model defines the geometry and parameters of the detector. Multiple detectors can share the same model (and this is in fact very common). Most required models are directly shipped with the framework.
\item An optional \textbf{electric field} in the sensitive device. Outside the sensitive device (and inside the sensitive device if no electric field is defined) the electric field is zero by definition.
\end{itemize}

The detector configuration is provided in the special detector configuration. \todo where should we explain the format of this configuration: in the core or in the module part \todo

\subsubsection{Coordinate frames}
The global frame has an arbitrary origin as there is nothing fixed in the world frame except the detectors which are defined relative to this arbitrary origin. Every detector also has a local coordinate system attached to it. The origin of this local coordinate system should NOT necessarily correspond with the geometric center of the sensitive device (and therefore the position and rotation center of the detector in the global frame). The origin of the local coordinate system is instead fixed to the center of the bottom-left collecting element (now referred as a pixel). 

The positive z-axis should point from the side of the sensor where the collection takes place upwards (normal to the bottom plane) to the top of the sensitive device. The x-axis should point in one of the arbitrary two directions in the plane where the pixel grid is defined. The y-axis should then be normal to both the x and the z-axis in such a way that a righ-handed coordinate system is constructed. 

\subsubsection{Detector models}
Different types of detector models can be used inside the framework. All models should extend from the DetectorModel base class which defines the minimum detector model defined in the framework:
\begin{itemize}
\item The coordinate of the rotation center (thus the specified position of the point in global frame) in the local frame.
\item The position of the bottom-left (minimum) of the sensor (thus the sensitive device) in the local frame. This determines the size of the border ring ('guard ring') around the pixel grid. 
\item The total size of the sensor (thus the top-right corner offset from the bottom-left minimum position).
\item The number of pixels in the sensor. Every pixel is basically an independent block replicated over the bottom of the sensor. This default to one if not overridden, which means the sensor has no replicated blocks.
\item The size of an individual pixel. The multiplication of the pixel size and the number of pixels should not exceed the sensor.
\end{itemize}

This standard detector model can be extended to provide a more extended geometry if required by certain (more advanced) modules. Currently the only supported option is the PixelDetectorModel (\todo should we change this name? \todo), which apart from the sensor can also include guards rings, bump bonds, a chip, a PCB and a cover layer. For more details about the different types of supported models and how to add your own new model one should look at section \ref{sec:adding_detector_model}.

Most of the standard detector models are shipped with the framework in the configuration format introduced in section \ref{sec:config_file_format}. Other models can however be used additionally. The conversion of these configuration files to detector models happens in the geometry construction module (\todo is this where it should be \todo). The models should be searched for in the directories given in the following order:
\begin{enumerate}
\item If defined, the paths in the \textit{models\_path} parameter provided to the reader module (overridden by the global configuration \textit{models\_path}) in the given order
\item The location where the models are installed to (see the MODEL\_DIRECTORY variable in section \ref{sec:cmake_config}). 
\item The standard data paths on the system as given by the environmental variable \$XDG\_DATA\_DIRS with the allpix (\todo get this directory from the build variable \todo) directory appended. The \$XDG\_DATA\_DIRS variable defaults to \textit{/usr/local/share/} (thus effectively /usr/local/share/allpix/) followed by \textit{/usr/share/} (effectively /usr/share/allpix/).
\end{enumerate}

\subsection{Passing Objects using Messages}
\label{sec:objects_messages}
Communication between modules happens almost solely through messages (only some information is shared through the internally used dependencies like Geant4). Most messages are templated instantiations of the Message class carrying a vector of objects. A typedef is typically added by the object to provide an alternative name for the message directly linking to the object it is carrying. The message system has a dispatching part and a receiver part. The dispatching module can specify an optional name for the message and the receiver can also optionally specify a name to listen to. The following rules are used to dispatch the message to their receivers:
\begin{enumerate}
\item If the message type (thus usually the object type) of the receiving module does not match the type of the sending module, the message will NEVER be dispatched to this receiving module.
\item If the module is a detector module and the dispatched message has a detector attached, the message will NEVER be dispatched to this module if the detector does NOT match the detector linked to the detector module.
\item If the name of the dispatched message is unspecified (or empty) the message will be dispatched to all modules (not excluded by earlier rules).
\item If the dispatched message has a specific name the message will only be dispatched to modules listening to exactly that name AND also to general listening modules that did not specify the name to receive (or set it to empty).
\end{enumerate}

\subsubsection{Methods to process messages}
The message system has multiple methods to process received messages. The first of these two are the most common methods and the third one should only rarely be used. The options are:
\begin{enumerate}
\item Bind a \textbf{single message} to a variable. This should usually be the preferred method as most modules only expect one message to arrive per event (as a module should typically send only one message containing the list of all the objects it should send).
\item Bind a \textbf{set of messages} to an array variable. This method should be used it the module can (and expects to) receive the same message multiple times (possibly because it wants to receive the same type of message for all detectors). 
\item Listen to a particular message type and execute a \textbf{listening function} as soon as an object is received. Should be used for more advanced strategies for fetching messages. The listening module should NEVER do any heavy work in the listening function as this is supposed to take place in the run() method instead.
\end{enumerate}

\subsubsection{Messenger flags}
Various flags can be added to the bind function and listening functions. The flags enable a particular behaviour of the framework (if the particular type of method supports the flag).
\begin{itemize}
\item \textbf{REQUIRED}: Specify that this message is required to be received. If the particular type of message is not received before it is time to execute the run function, the run is automatically skipped by the framework. This can be used to ignore actions that would be useless without received messages, like propagating charges without any deposits.
\item \textbf{NO\_RESET}: Messages are by default automatically resetted after the run function executes to prevent older messages from previous runs to appear again. This behaviour can be disabled by setting this flag (this does not have any effect for listening functions). Setting this flag for single bound messages (without ALLOW\_OVERWRITE) would causes an exception to be raised if the message is overwritten in a later event.
\item \textbf{ALLOW\_OVERWRITE}: By default an exception is automatically raised if a single bound message is overwritten (thus set multiple times instead of once). This flag prevents this behaviour, it is only useful for single bound variables. 
\end{itemize}

\subsubsection{Object types}
All supported objects that can be transferred between modules are shipped with the framework in the Objects library. This list of objects currently consists of the following:
\begin{itemize}
\item \textbf{DepositedCharge}: Set of charges at a specific position in the sensor of a detector.
\item \textbf{PropagatedCharge}: Charge at a specific position after propagation \todo THIS OBJECT WILL BE REMOVED? \todo.
\item \textbf{PixelCharge}: Set of charges at a particular pixel in the pixel grid.
\end{itemize}

\todo This should be a separate section \todo

\subsection{Logging and other Utilities}
\label{sec:logging_utilities}
The \apsq framework provides a set of utilities that can be split up in two type of utilies:
\begin{itemize}
\item A few utilities to extend the Standard Template Library (STL) \needref. These are provided either to simplify access to the STL (like the random seed generator utility) or to provide functionality the C++14 standard lacks (like filesystem support). The referred utilities are used internally in the framework and only shortly discussed. The utilities falling in this category are the random seed generator (see section \ref{sec:random_generator}), the filesystem functions (see section \ref{sec:filesystem}) and the string utilies (see section \ref{sec:string_utilities}).
\item Two utilities to improve the usability of the framework. One of these is a flexible and easy-to-use logging system, introduced below in section \ref{sec:logger} and revisited in section \ref{sec:error_reporting_exceptions}. The other is an easy-to-use framework for units that supports converting arbitrary combinations of units to an independent number which can  be used transparently through the framework. It will be discussed in more detail in section \ref{sec:unit_system}.
\end{itemize}

\subsubsection{Logger}
\label{sec:logger}
The logger is build to handle input/output in the same way as std::cin and std::cout do. This system is both very flexible and easy to read. It is globally configured, thus there exists only one logger and there no ocal loggers in for example a module. To send a message to the logger at a level of \textbf{LEVEL}, the following can be used:
\begin{minted}[frame=single,framesep=3pt,breaklines=true,tabsize=2,linenos]{c++}
LOG(LEVEL) << "this is an example message with data " << 1 << 2.0;
\end{minted}
Note that a newline is added at the end of every log message. Multi-line log messages can also be used: the logger will automatically align the new lines under the message itself and will leave the header space empty on the new lines.

The various logging options for \textbf{LEVEL} are as follows:
\begin{itemize}
\item \textbf{FATAL}: Indicates a fatal error that will lead to direct termination of the application. Typically only used by the exception catchers in the main executable as exceptions should be the only way to request the framework to terminate.
\item \textbf{ERROR}: Severe error that should never happen during a normal well-behaving simulation run. Frequently leads to a fatal error and can be used to provide extra information not associated to the exception.
\item \textbf{WARNING}: Indicate conditions that should not happen normally and can lead to unexpected output results. The framework can however typically continue without problems after a warning.
\item \textbf{INFO}: Informatic messages about the process of the simulation
\item \textbf{DEBUG}: In-depth details about what the framework or a particular module is doing. Can be used to get more detailed information about the modules behaviour or to debug modules.
\end{itemize}
\todo More logging levels should be added here and the framework should be improved to better adhere to these levels. \todo

The logger can display the log messages in several formats. Currently the following formats are supported:
\begin{itemize}
\item \textbf{SHORT}: Displays the data in a short form. Includes only the first character of the log level followed by the section header and the message.
\item \textbf{DEFAULT}: The default format. Displays the date, log level, section header and the message itself. 
\item \textbf{DEBUG}: Detailed logging level. Besides the information above, it also shows the file and line where the log message came from. This can help in debugging the modules.
\end{itemize}

\subsubsection{Unit system}
\label{sec:unit_system}
In physics (and therefore HEP as well) units are used a lot. Having a separate C++ type for all different kind of units would be too cumbersome for a lot of operations. Therefore the units are stored in standard C++ types in a pre-defined unit and all code in the framework uses the same unit. In various configuration files it is however very useful to give quantities in another unit. For logging it is also useful to display data in another unit.

The unit system allows adding, getting and converting units. It is a global system, thus all units should be equal through the whole framework. Examples of using the unit system are the given below:
\begin{minted}[frame=single,framesep=3pt,breaklines=true,tabsize=2,linenos]{c++}
// define the standard length unit and an auxiliary unit
Units::add("mm", 1); 
Units::add("m", 1e3); 
// define the standard time unit
Units::add("ns", 1); 
// get the units given in m/ns in the defined framework unit mm/ns
Units::get(1, "m/ns"); 
// get the framework unit of mm/ns in m/ns for logging etc
Units::convert(1, "m/ns");
\end{minted}

More details about the unit system used within AllPix can be found in section \needref.

\subsubsection{Internal utilities}
\paragraph{Filesystem}
\label{sec:filesystem}
Provide functions to convert relative to absolute canonical paths, to go through all files in a directory and to create new directories. These functions should be replaced by the C++17 filesystem API\needcite as soon as the framework minimum standard is updated to C++17.

\paragraph{String utilies}
\label{sec:string_utilities}
The STL only provides string conversions for standard types using stringstream and std::to\_string. It does not allow to parse strings encapsulated in pairs of \texttt{"} characters. Furthermore it does not provide wide flexibility to add custom conversions for other external types in both ways. The \apsq to\_string and from\_string provide the flexibility to do these conversions: it it extensively used in the configuration system. The tools contain extensions to allow automatic conversions for ROOT and Geant4 types. Also utilities to trim and split strings are included as they are not yet part of the STL.

\paragraph{Random seed generator}
\label{sec:random_generator}
Generating random number with a high-level of entropy is very important for running all Monte-Carlo processes in high-energy physics. Good random generators are already included in both the standard library and various external libraries. Seeding those random generators with enough entropy or seeding them with a predefined seed instead (to allow rerunning a certain simulation) should however be a framework task. This small utility library provides the required method to return a correct seed to all random generators.

\subsection{Error Reporting and Exceptions}
\label{sec:error_reporting_exceptions}
